%!TEX root = ../RobotOblig3.tex
%Oppgave2.tex

Den odometriske lokaliseringen ble utført ved å bruke eksakt integrasjon. Ny \(x\) og ny \(y\) ble regnet ut med
\begin{equation}
\begin{array}{ll}
x_{k+1} = x_k + \frac{v_k}{\omega_k}(sin\theta_{k+1} - sin\theta_k) \\
y_{k+1} = y_k - \frac{v_k}{\omega_k}(cos\theta_{k+1} - cos\theta_k) \\
\theta_{k+1} = \theta_k + \omega_kT_s
\end{array}
\end{equation}

Her er \(T_s\) samplingstiden, som i denne oppgaven var satt til 50 ms. Den eksakte integrasjonen er ikke definert for \(\omega = 0\), og vi måtte derfor ta i bruk integrasjonsmetoder fra Eulermetoden eller Runge Kutta. For \(\omega = 0\) er ny \(x\) og ny \(y\) definert som

\begin{equation}
\begin{array}{ll}
x_{k+1} = x_k + v_kT_scos\theta_k \\
y_{k+1} = y_k + v_kT_ssin\theta_k
\end{array}
\end{equation}

Den odometriske lokaliseringsnoden ble implementert til å ta inn mål fra hjulencoderne (gitt i m/s) og regne ut lokaliseringen ved å bruke følgende kode:

\begin{lstlisting}[frame=single]
void updatePosition(rover1_motordriver::Velocity vel){

        float diffVel = (vel.right_vel - vel.left_vel);
        float sumVel = (vel.right_vel + vel.left_vel);
        float omega = (diffVel)/m_diameter;
        float oldTheta = m_theta;

        m_theta = m_theta + omega*m_timeInterval;
        m_theta = fmod(m_theta, 2*3.14);
        if(m_theta > 3.14) m_theta = m_theta-2*3.14;
        if(m_theta < -3.14) m_theta + 2*3.14;

        m_velocity = sumVel/2;

        if (omega != 0){
            m_position.x = m_position.x + (m_velocity/omega)*(sin(m_theta) - sin(oldTheta));
            m_position.y = m_position.y - (m_velocity/omega)*(cos(m_theta) - cos(oldTheta));
        }
        else{
            m_position.x = m_position.x + m_velocity*cos(oldTheta);
            m_position.y = m_position.y + m_velocity*sin(oldTheta);
        }

    }

\end{lstlisting}
