%%%%%%%%%%%%%%%%%%%%%%%%%%%%%%%%%%%%%%%%%

%----------------------------------------------------------------------------------------
%	PACKAGES AND OTHER DOCUMENT CONFIGURATIONS
%----------------------------------------------------------------------------------------

\documentclass[a4paper,10pt]{article}

\usepackage[utf8]{inputenc} % Required for inputting international characters
\usepackage[T1]{fontenc} % Output font encoding for international characters
% !TEX encoding = UTF-8 Unicode
%\usepackage[utf8]{inputenc}
\usepackage{pdfpages}
\usepackage{mathtools}
\usepackage{amsmath}
\usepackage{fullpage}
\usepackage{bm}
\usepackage{float}
\usepackage{gensymb}
\graphicspath{./dir/} 
\usepackage{mathpazo} % Palatino font

\begin{document}

%----------------------------------------------------------------------------------------
%	TITLE PAGE
%----------------------------------------------------------------------------------------

\begin{titlepage} % Suppresses displaying the page number on the title page and the subsequent page counts as page 1
	\newcommand{\HRule}{\rule{\linewidth}{0.5mm}} % Defines a new command for horizontal lines, change thickness here
	
	\center % Centre everything on the page
	
	%------------------------------------------------
	%	Headings
	%------------------------------------------------
	
	\textsc{\LARGE University of Oslo}\\[1.5cm] % Main heading such as the name of your university/college
	
	\textsc{\Large Control of Mobile Robots}\\[0.5cm] % Major heading such as course name
	
	\textsc{\large UNIK4490}\\[0.5cm] % Minor heading such as course title
	
	%------------------------------------------------
	%	Title
	%------------------------------------------------
	
	\HRule\\[0.4cm]
	
	{\huge\bfseries An Unnecessarily Extra Long Convoluted Academic Title That Makes Little Sense}\\[0.4cm] % Title of your document
	
	\HRule\\[1.5cm]
	
	%------------------------------------------------
	%	Author(s)
	%------------------------------------------------
	
	\begin{minipage}{0.4\textwidth}
		\begin{flushleft}
			\large
			\textit{Authors}\\
			Daniel \textsc{Sander Isaksen}, Eirik \textsc{Kvalheim} and Torgrim \textsc{R. Næss}
		\end{flushleft}
	\end{minipage}
	~
	\begin{minipage}{0.4\textwidth}
		\begin{flushright}
			\large
			\textit{Supervisors}\\
			Dr. Kim \textsc{Mathiassen} and Magnus \textsc{Baksaas}
		\end{flushright}
	\end{minipage}
	
	% If you don't want a supervisor, uncomment the two lines below and comment the code above
	%{\large\textit{Author}}\\
	%John \textsc{Smith} % Your name
	
	%------------------------------------------------
	%	Date
	%------------------------------------------------
	
	\vfill\vfill\vfill % Position the date 3/4 down the remaining page
	
	{\large\today} % Date, change the \today to a set date if you want to be precise
	
	%------------------------------------------------
	%	Logo
	%------------------------------------------------
	
	%\vfill\vfill
	%\includegraphics[width=0.2\textwidth]{placeholder.jpg}\\[1cm] % Include a department/university logo - this will require the graphicx package
	 
	%----------------------------------------------------------------------------------------
	
	\vfill % Push the date up 1/4 of the remaining page
	
\end{titlepage}

%----------------------------------------------------------------------------------------


\renewcommand{\labelenumi}{\alph{enumi})}


\section*{Some Heading}

\begin{figure}[H]
\centering
 \includegraphics[width=0.5\textwidth]{rover_pic.png}
 \caption{4 by 4 Rover}
 \label{fig:1}
\end{figure}

The main task of the project was to make the robot move to a desired pose, and this was divided into four subtasks/stages as listed below the following summary:\\  
\textit{A significant portion of the project time was spent on reverse engineering the robot to better understand the system before we could begin implementing our own solutions.  
We decided to keep the original PID motor control code and write code for odometric localization and posture regulation. 
Most of the work was done in collaboration with the other group, 
as we were working on the same robot while trying to get the system up and running. \\
(Note! One important subtask that is not mentioned in the list was to create a system of ROS nodes, for which we had to learn about ROS.)}

\begin{figure}[H]
\centering
 \includegraphics[width=0.5\textwidth]{blokkskjema}
 \caption{Block diagram of the system.}
 \label{fig:2}
\end{figure}

\begin{itemize}
	\item[1.]Implement motor control for each wheel \\
		We started our project by running and reverse engineering the mobile robot together with the other group. 
		As we were not familiar with the system and since there were no README or comments in the code, the challenges for reverse engineering the motor controller was: 
		to login, to run robot, understand the controller and deduce the reason behind the constants in the controller.
		The implementation of motor control for each wheel was already avaiable in the robots source. 
		
	\item[2.]Implement odometric localization
	\item[3.]Implement posture regulation motion control \\
		In general the posture regulation controller takes in the configuration variables, $q = [x, y, \theta]^{T}$, 
		and outputs v and $\omega$.
		It is assumed that the desired variables are $q_{d} = [0, 0, 0]^{T}$ and the error from $q_{d}$ is represented by following variables:
		\[
			\rho = \sqrt{x^{2} + y^{2}} 
		\]\[
			\gamma = Atan2(y, x) - \theta + \pi
		\]\[
			\delta = \gamma + \theta
		\]
		Where $\rho = ||\vec{e_{p}}||$ is the distance between current point $(x, y)$ and desired point $(0, 0)$, 
		$\gamma$ is the angle between $\vec{e_{p}}$	and the sagittal axis of the vehicle and $\delta$ is the axis between $\vec{e_{p}}$ and the x-axis.	
		v and $\omega$ are found by:
		\begin{equation}
			v = k_{1}\rho cos(\gamma)
		\end{equation}		
		\begin{equation}
			\omega = k_{2}\gamma + k_{1}\frac{sin(\gamma)cos(\gamma)}{\gamma}(\gamma + k_{3}\delta)
		\end{equation}
		
		In our implementation of the controller we get $\vec{q}$ from the odometric module and output $\omega_{R}$ and $\omega_{L}$ to the motor controller.
		Equations for $\omega_{R}$ and $\omega_{L}$ expressed by error variables $\rho$, $\gamma$ and $\delta$, 
		by setting the following equations (3) and (4) equal to equations (1) and (2) respectively, 
		\begin{equation}
			v = \frac{r(\omega_{R} + \omega_{L})}{2}
		\end{equation}
		\begin{equation}
			\omega = \frac{r(\omega_{R} - \omega_{L})}{d}
		\end{equation}
		and then solve for $\omega_{R}$ and $\omega_{L}$ by the inserting method. This yields:		
		\begin{equation}
			\omega_{R} = \frac{2k_{1}\rho cos(\gamma)}{2r} + \frac{dk{2}\gamma}{2r} + \frac{dsin(\gamma)cos(\gamma)(\gamma + k_{3}\delta)}{2r\gamma}
		\end{equation}
		\begin{equation}
			\omega_{L} = \frac{2k_{1}\rho cos(\gamma)}{2r} - \frac{dk_{2}\gamma}{2r} - \frac{dsin(\gamma)cos(\gamma)(\gamma + k_{3}\delta)}{2r\gamma}
		\end{equation}
		
	\item[4.]Move the robot to a pose
\end{itemize}


\end{document}
